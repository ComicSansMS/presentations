\documentclass{beamer}

\usepackage[utf8]{inputenc}

\usepackage{color}
\usepackage{listings}
\usepackage{tikz}
\usepackage{hyperref}

\usetheme{Rochester}
\usecolortheme{beaver}

\usetikzlibrary{decorations.text,calc}

\newcommand{\tikzmark}[2]{%
     \tikz[overlay,remember picture] \node[text=black,
           inner sep=2pt] (#1) {#2};}
\newcommand{\VerticalShiftForArrows}{0.0em,+2.75ex}%
\newcommand{\HorizontalShiftForBar}{2.0em,+0.0ex}%
\newcommand{\Stub}{0.0em,-0.6ex}%
\newcommand{\DrawSideBar}[3][]{%
        \coordinate (top left)  at ($(#2)       +(\HorizontalShiftForBar)$);
        \coordinate (start)     at ($(top left) +(\Stub)$);
        \coordinate (top right) at ($(#3)       +(\HorizontalShiftForBar)$);
        \coordinate (end)       at ($(top right)+(\Stub)$);
        \draw [#1] (start) -- (top left) -- (top right) -- (end);
}%
\newcommand{\DrawArrows}[3][]{%
    \coordinate (Start Mid) at ($(#2Left)!0.5!(#2Right) + (\VerticalShiftForArrows)$);
    \coordinate (End Mid)   at ($(#3Left)!0.5!(#3Right) + (\VerticalShiftForArrows)$);

    \draw[-stealth, thick, #1] (Start Mid) to (End Mid);
}


\newif\iftransitions
 \transitionstrue


\title{A Quick Tour of Thread Progress Guarantees}
\author{Andreas Weis}
\institute{BMW AG}
\date{cppcon 2017}
\titlegraphic{\includegraphics[height=.25\textheight]{resources/cppcon.png}}


\begin{document}

\frame{\titlepage}

\begin{frame}[fragile]
  \frametitle{Let's talk about concurrency}

  \begin{itemize}
\iftransitions \pause \fi
    \item Concurrent access to shared data needs to be synchronized.
\iftransitions \pause \fi
    \item We typically use \texttt{std::mutex} for that.
\iftransitions \pause \fi
    \item A thread attempting to lock an unavailable mutex will \emph{block}.
\iftransitions \pause \fi
    \item Hopefully not forever.
    \begin{center}
        \includegraphics[height=0.4\textheight]{resources/skeleton.jpg}
    \end{center}
  \end{itemize}
\end{frame}

\begin{frame}[fragile]
  \frametitle{Deadlock-Freedom \& Starvation-Freedom}

  \begin{itemize}

    \item An algorithm is \emph{deadlock-free}, if of all concurrent requests at least one succeeds.

    \iftransitions \pause \fi

    \item An algorithm is \emph{starvation-free}, if all concurrent requests succeed eventually.

  \end{itemize}
\end{frame}

\begin{frame}[fragile]
  \frametitle{So we're good now, right?}

  Both categories depend on...

  \iftransitions \pause \fi

  \begin{center}
  \includegraphics[height=0.7\textheight]{resources/robot-devil-scheduler.jpg}
  \end{center}
\end{frame}


\begin{frame}[fragile]

\frametitle{Beware the evil scheduler!}

\begin{lstlisting}[language=C++,basicstyle=\ttfamily,keywordstyle=\color{blue},commentstyle=\color{teal}\itshape,showstringspaces=false,escapeinside={(*@}{@*)}]
std::mutex mtx;

// [...]
{
    std::lock_guard lk(mtx);
(*@\tikzmark{aLeft}{}@*)
    // do critical stuff
    // [...]
(*@\tikzmark{aRight}{}@*)
}(*@
\iftransitions \pause \fi
\begin{tikzpicture}[overlay,remember picture]
    \DrawSideBar[-, red, thick]{aLeft.south}{aRight.south}
\end{tikzpicture}
@*)\end{lstlisting}

\begin{itemize}
    \item By trapping one thread inside the critical section, the scheduler can keep \emph{all threads} from making progress.
\end{itemize}

\end{frame}

\begin{frame}[fragile]

\frametitle{So, don't use mutexes and you're good!}

Right?

\end{frame}

\begin{frame}[fragile]

\frametitle{Nah, not really...}

\begin{lstlisting}[language=C++,basicstyle=\ttfamily,keywordstyle=\color{blue},commentstyle=\color{teal}\itshape,showstringspaces=false,escapeinside={(*@}{@*)}]

\end{lstlisting}

\end{frame}


\begin{frame}
  \frametitle{Thanks for your attention.}
\end{frame}


\end{document}
